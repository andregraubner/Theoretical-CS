\documentclass[a4paper]{article}

\usepackage{fullpage}
\usepackage{parskip}
\usepackage{tikz}
\usepackage{amsmath}
\usepackage{amsfonts}
\usepackage{amssymb}
\usepackage{hyperref}
\usepackage[utf8]{inputenc}

\newtheorem{theorem}{Theorem}
\newtheorem{lemma}{Lemma}

\title{Serie 1}
\author{André Graubner, Leonard von Kleist, Lukas Walker}
\date{2018/09/27}

\begin{document} 

\maketitle

\section{Aufgabe 1a}
	Um die maximale Anzahl von Teilworten zu haben,
	muss das Wort jedes Zeichen genau einmal enthalten, 
	da sonst mindestens das Teilwort welches nur aus 
	dem doppelt vorkommenden Zeichen besteht doppelt 
	verwendet wird. Es gilt:

	\begin{lemma}
		Sei $w_{max}$ ein Wort von Länge m und eine Permutation von $m$ Zeichen aus $\Sigma$. Dann gilt:
		\[\Bigg\lvert \Big\{x \mid x \text{ ist Teilwort von } w_{max} \Big\} \Bigg\lvert = 
		\bigg(\sum_{i=1}^{m} i\bigg) + 1 = \frac{m^2 + m}{2} + 1 \]

		Beweis:
		Jedes Teilwort (ausser $\lambda$) lässt sich als Tupel $(a,b)$ eindeutig darstellen,
		wobei $a$ die Startposition und $b$ die Endposition des Teilwortes
		innerhalb von $w_{max}$ bezeichnet. Hier muss gelten:
		\[a, b \in \mathbb{N} \text{ mit } 1 \leq a \leq b \leq m\]
		Von solchen Tupeln gibt es $\frac{m^2 + m}{2}$, doch wir haben $\lambda$
		noch nicht mitgezählt, das von solchen Tupeln nicht beachtet wird.
		Damit kommen wir auf die oben genannte Summe.
	\end{lemma}

\section{Aufgabe 1b}
	Die Gesamtzahl aller Wörter w mit $\lvert x \rvert = n$ ist $3^n$.
	Von dieser Zahl müssen wir nun die Zahl der Wörter abziehen, die
	mindestens eines der Zeichen nicht enthalten. Für 2 Zeichen aus 
	$\Sigma$ kann ich mit diesen genau $2^n$ Wörter der Länge $n$ bilden.
	Dabei gibt es 3 mögliche Kombinationen der Zeichen (siehe $\binom{3}{2}$).
	Wir zählen jedoch genau die Wörter doppelt, die aus genau einem
	Zeichen gebildet werden können. Wenn wir diese wieder abziehen erhalten wir:
	\[\big\{x \mid \text{$x$ enthält $a$, $b$ und $c$} \big\} = 3^n - (3 \cdot 2^n) + 6\]

\section{Aufgabe 2a}
	Wir verwenden für den Beweis eine Fallunterscheidung.

	Fall 1: Wenn $u = v = \lambda$, so gilt die Gleichheit trivialerweise.
	\[(uv)^R = (\lambda\lambda)^R = \lambda^R = \lambda^R \lambda = \lambda^R \lambda^R = v^R u^R\]

	Fall 2: Wenn $u = \lambda \neq v$, so gilt:
	\[(uv)^R = (\lambda v)^R = v^R = v^R \lambda = v^R \lambda^R = v^R u^R\]

	Fall 3: Wenn $v = \lambda \neq u$, so gilt:
	\[(uv)^R = (u \lambda)^R = u^R = \lambda u^R = \lambda^R u^R = v^R u^R\]
	
	Fall 4: Weder $u$ noch $v$ sind $\lambda$.
	Dann lassen sich $u$, $v$ und $uv$ folgendermassen eindeutig darstellen:
	\[u = u_{1}u_{2}...u_{n} \text{ und } v = v_{1}v_{2}...v_{m} \]
	Also gilt per Definition:
	\[u^R = u_{n}u_{n-1}...u_{1} \text{ und } v^R = v_{m}v_{m-1}...v_{1}\]
	Daraus folgt:
	\[(uv)^R = (u_{1}u_{2}...u_{n}v_{1}v_{2}...v_{m})^R 
	= v_{m}v_{m-1}...v_{1}u_{n}u_{n-1}...u_{1} = v^R u^R\]

	Da dies die einzigen möglichen Fälle sind, gilt die Gleichheit für beliebige $u$ und $v$. $\blacksquare$

\section{Aufgabe 2b}
	Sei $L_{1} = \Sigma^*$, $ L_{2} = L_{\lambda}$ und $ L_{3} = \big\{ 0 \big\}$.

	Es gilt:
	\[L_{1}(L_{2} \cap L_{3}) = L_{1}L_{\emptyset} = L_{\emptyset}\]
	\[L_{1}L_{2} \cap L_{1}L_{3} = \Sigma^* \cap (\Sigma^* - \big\{ \lambda \big\}) = \Sigma^* - \big\{ \lambda \big\}\]

	Hierbei ist insbesondere $L_{\emptyset}$ endlich (leer) und $\Sigma^* - \big\{ \lambda \big\}$ unendlich. $\blacksquare$

\section{Aufgabe 3}
	(a): Sei $\Sigma = \big\{ 0 \big\}$, $L_{1} = \Sigma^{< k}$ und $L_{2} = \big\{ \lambda, 0 \big\}$.
	Für jede mögliche Konkatenation in $L_{1}L_{2}$ ist das Ergebnis bereits in $L_{1}$, ausser 
	für $0^{k-1} \cdot 0 = 0^k$.

	(b): Sei $\Sigma = \big\{ 0 \big\}$, $L_{1} = \Sigma^{< k}$ und $L_{2} = \big\{ \lambda, 0, 00, 000, 0000, 00000 \big\}$.
	Wieder ist für jede mögliche Konkatenation in $L_{1}L_{2}$ das Ergebnis bereits in $L_{1}$, ausser
	für $0^{k-1} \cdot x$ mit $x \in L_{2} - \big\{ \lambda \big\}$.
	
	(c,d): Da eine Lösung für Teilaufgabe (d) auch eine Lösung für Teilaufgabe (c) ist, lösen wir beide auf einmal.
	Dafür abstrahieren wir das Problem zunächst: Da wir ein Alphabet mit einem Zeichen wählen sollen, ist es vollkommen
	egal, welches wir hier wählen. Der Einfachheit halber wählen wir jedoch $\Sigma = \big\{ 0 \big\}$.
	Um nun für beliebige Sprachen $L_{1}$ und $L_{2}$ die Eigenschaft $\lvert L_{1}L_{2} \rvert = \lvert L_{1} \rvert \cdot \lvert L_{2} \rvert$
	zu haben muss gelten, dass jede Kombination von Elementen aus den Sprachen einzigartig ist.
	Wir können die Elemente aus $L_{1}$ und $L_{2}$ hierbei sinnvollerweise als natürliche Zahlen darstellen (wobei die Zahl $n$ dem Wort $0^n$ entspricht).
	Das Problem ist nun lediglich, zwei Mengen $A$ und $B$ von natürlichen Zahlen zu finden, sodass die Summen von allen Paaren $(a,b)$ einzigartig sind,
	weil die Konkatenation $0^n \cdot 0^m = 0^{n+m}$ ist.
\end{document}
