\documentclass[a4paper]{article}

\usepackage{fullpage}
\usepackage{parskip}
\usepackage{tikz}
\usepackage{amsmath}
\usepackage{amsfonts}
\usepackage{hyperref}

\newtheorem{theorem}{Theorem}
\newtheorem{lemma}{Lemma}

\title{Serie 1}
\author{André Graubner, Leonard von Kleist, Lukas Walker}
\date{2018/09/27}

\begin{document} 

\maketitle

\section{Aufgabe 1a}
	Um die maximale Anzahl von Teilworten zu haben,
	muss $m = n$ gelten und das Wort jedes Zeichen
	genau einmal enthalten, da sonst mindestens das
	Teilwort welches nur aus dem doppelt vorkommenden
	Zeichen besteht doppelt verwendet wird. Es gilt:

	\begin{lemma}
		Sei $w_{max}$ eine Permutation der $n$ Zeichen in $\Sigma$. Dann gilt:
		\[\Bigg\lvert \Big\{x \mid x \text{ ist Teilwort von } w_{max} \Big\} \Bigg\lvert = 
		\bigg(\sum_{i=1}^{n} i\bigg) + 1 = \frac{n^2 + n}{2} + 1 \]

		Beweis:
		Jedes Teilwort (ausser $\lambda$) laesst sich als Tupel $(a,b)$ eindeutig darstellen,
		wobei $a$ die Startposition und $b$ die Endposition des Teilwortes
		innerhalb von $w_{max}$ bezeichnet. Hier muss gelten:
		\[a, b \in \mathbb{N} \text{ mit } 1 \leq a \leq b \leq n\]
		Von solchen Tupeln gibt es $\frac{n^2 + n}{2}$, doch wir haben $\lambda$
		noch nicht mitgezählt, das von solchen Tupeln nicht beachtet wird.
		Damit kommen wir auf die oben genannte Summe.
	\end{lemma}

\end{document}
