\documentclass[a4paper]{article}

\usepackage{fullpage}
\usepackage{parskip}
\usepackage{tikz}
\usepackage{amsmath}
\usepackage{hyperref}

\newtheorem{theorem}{Theorem}
\newtheorem{lemma}{Lemma}

\title{Übung 1}
\author{André Graubner, Leonard von Kleist, Lukas Walker}
\date{2018/09/27}

\begin{document} 

\maketitle

\section{Aufgabe 1a}
	Um die maximale Anzahl von Teilworten zu haben,
	muss $m = n$ gelten und das Wort jedes Zeichen
	genau einmal enthalten. Dann gilt:

	\begin{lemma}
		Sei $w_{max}$ eine Permutation der $n$ Zeichen in $\sigma$.
		\[R \text{ is non-trivial} \iff 0 \neq 1\]
	\end{lemma}

\end{document}
