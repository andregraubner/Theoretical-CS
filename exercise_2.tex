\documentclass[a4paper]{article}

\usepackage{fullpage}
\usepackage{parskip}
\usepackage{tikz}
\usepackage{amsmath}
\usepackage{amsfonts}
\usepackage{amssymb}
\usepackage{hyperref}
\usepackage[utf8]{inputenc}

\newtheorem{theorem}{Theorem}
\newtheorem{lemma}{Lemma}

\title{Serie 2}
\author{André Graubner, Lukas Walker, Leonard von Kleist}
\date{28. September 2018}

\begin{document} 

\maketitle

\section{Aufgabe 4}
	
	Sei $L$ eine unendliche Sprache über einem Alphabet $\Sigma$.
	
	Richtung 1 ($\Rightarrow$):
	Sei $L$ rekursiv.
	Das heisst per Definition, dass ein Algorithmus $A_{erk}$ existiert, der $L$ erkennt.
	Wir konstruieren einen Aufzählungsalgorithmus (mit Eingabe $n$) für $L$:
	Iteriere in kanonischer Reihenfolge durch $\Sigma^*$ und überprüfe mit $A_{erk}$ für 
	jedes Element, ob es $\in L$ ist. Falls ja, gebe es aus. Fahre so lange fort, bis wir
	$n$ Elemente ausgegeben haben.

	Richtung 2 ($\Leftarrow$):
	Sei $A_{enum}$ (mit Eingabe $n$) ein Aufzählungsalgorithmus für $L$.
	Wir konstruieren einen Algorithmus (mit Eingabe $x$), der $L$ erkennt:
	Bezeichne $k$ die Stelle, an der das Wort $x$ kanonisch in $\Sigma^*$ steht.
	Rufe $A_{enum}$ mit Eingabe $k$ auf. Wenn eine der von $A_{enum}$ ausgegebenen
	Wörter $x$ ist, so ist $x \in L$, also geben wir 1 (bzw. "ja", etc) aus, sonst 0.

\section{Aufgabe 5}

	Im Intervall $\mathopen[2^n, 2^{n+1}-1 \mathopen]$ gibt es $2^{n+1} - 2^n - 1$ verschiedene Zahlen.
	Wir betrachten jeweils lediglich die kürzesten Maschinencodes, die diese Zahlen generieren.
	Weiterhin ist klar, dass diese Codes für zwei verschiedene Zahlen unterschiedlich sein müssen. 
	Es reicht also zu zeigen, dass von $2^{n+1} - 2^n - 1$ paarweise unterschiedlichen binären Strings
	für jedes $i < n$ mindestens $2^n - 2^{n-i}$ die Länge $n - i$ haben.

	Die Anzahl der (nichtleeren) Bitstrings von Länge kleiner als $n - i$ ist 
	\[\sum_{k=1}^{n-i-1} 2^k = 2^{n-i} - 2\]
	Wenn wir diese Anzahl nun jedoch von der Menge an Zahlen in unserem Intervall abziehen, sehen wir:
	\[2^{n+1} - 2^{n} - 1 - 2^{n-i} + 2 = 2^{n+1} - 2^{n} + 1 - 2^{n-i} \geq 2^n - 2^{n-i}\]
	Also gilt für mindestens $2^n - 2^{n-i}$ der Zahlen $K(x) \geq n-i$.

\end{document}
