\documentclass[a4paper]{article}

\usepackage{fullpage}
\usepackage{parskip}
\usepackage{tikz}
\usepackage{fouriernc}
\usepackage{amsmath}
\usepackage{amsfonts}
\usepackage{amssymb}
\usepackage{hyperref}
\usepackage[utf8]{inputenc}

\newtheorem{theorem}{Theorem}
\newtheorem{lemma}{Lemma}

\title{Serie 2}
\author{André Graubner, Lukas Walker, Leonard von Kleist}
\date{28. September 2018}

\begin{document} 

\maketitle

\section{Aufgabe 4}
	
	Sei $L$ eine unendliche Sprache über einem Alphabet $\Sigma$.
	
	Richtung 1 ($\Rightarrow$):
	Sei $L$ rekursiv.
	Das heisst per Definition, dass ein Algorithmus $A_{erk}$ existiert, der $L$ erkennt.
	Wir konstruieren einen Aufzählungsalgorithmus (mit Eingabe $n$) für $L$:
	Iteriere in kanonischer Reihenfolge durch $\Sigma^*$ und überprüfe mit $A_{erk}$ für 
	jedes Element, ob es $\in L$ ist. Falls ja, gebe es aus. Fahre so lange fort, bis wir
	$n$ Elemente ausgegeben haben.

	Richtung 2 ($\Leftarrow$):
	Sei $A_{enum}$ (mit Eingabe $n$) ein Aufzählungsalgorithmus für $L$.
	Wir konstruieren einen Algorithmus (mit Eingabe $x$), der $L$ erkennt:
	Bezeichne $k$ die Stelle, an der das Wort $x$ kanonisch in $\Sigma^*$ steht.
	Rufe $A_{enum}$ mit Eingabe $k$ auf. Wenn eines der von $A_{enum}$ ausgegebenen
	Wörter $x$ ist, so ist $x \in L$, also geben wir 1 (bzw. "ja", etc) aus, sonst 0.

\section{Aufgabe 5}

	Im Intervall $\mathopen[2^n, 2^{n+1}-1 \mathopen]$ gibt es $2^n$ verschiedene Zahlen.
	Wir betrachten jeweils lediglich die kürzesten Maschinencodes, die diese Zahlen generieren.
	Weiterhin ist klar, dass diese Codes für zwei verschiedene Zahlen unterschiedlich sein müssen. 
	Es reicht also zu zeigen, dass von $2^n$ paarweise unterschiedlichen binären Strings
	für jedes $i < n$ mindestens $2^n - 2^{n-i}$ die Länge $n - i$ haben.

	Die Anzahl der (nichtleeren) Bitstrings von Länge kleiner als $n - i$ ist 
	\[\sum_{k=1}^{n-i-1} 2^k = 2^{n-i} - 2\]
	Wenn wir diese Anzahl nun jedoch von der Menge an Zahlen in unserem Intervall abziehen, sehen wir:
	\[2^n - (2^{n-i} - 2) = 2^n - 2^{n-i} + 2 \geq 2^n - 2^{n-i}\]
	Also gilt für mindestens $2^n - 2^{n-i}$ der Zahlen $K(x) \geq n-i$.

\section{Aufgabe 6a}

	Mit einem Programm der Form "Print "0" 8 * $n^4$ times" mit der Eingabe $n$ können wir jedes $w_n$ generieren.
	Hierbei hat alles in dem Programm ausser $n$ eine konstante Länge (wir können sogar das erste Bit von $n$ ignorieren, weil es sowieso eine 1 ist).

	Demnach gilt:
	\[K(w_n) \leq \left \lceil{\log{n}}\right \rceil + c \]

	Um die Komplexität nun im Verhältnis zu $w_n$ anzugeben, müssen wir also nun die Zahl $n$ als Funktion von der Länge von $w_n$ angeben.
	Hierbei gilt:

	\[\lvert w_n \rvert = 1 * 2 * 2 * 2 * n^4 = 8 * n^4 \Rightarrow n = \frac{\sqrt[4]{w_n}}{\sqrt{2}}\]

	Also ist eine obere Schranke für $K(w_n)$:

	\[K(w_n) \leq \left \lceil{\log{\frac{\sqrt[4]{w_n}}{\sqrt{2}}}}\right \rceil\]

\section{Aufgabe 6b}

	Sei $y_i = 2^{2^{n_{i}+2}}$ für ein $n_i \in \mathbb{N}_{>0}$. Dann gibt es ein Programm, welches $y_i$ aus $n_i$ konstruiert. Es gilt also:

	\[K(y_i) \leq \left \lceil{log{n_i}}\right \rceil + c \]

	Hierbei ist zu beachten, dass $n$ in folgendem Verhältnis zu $y_i$ steht:
	
	\[n_{i} = \log\log\sqrt[4]{y_i} \]

	Daraus folgt dass für alle $y_i$ gilt:

	\[K(y_i) \leq \left \lceil{\log\log\log\sqrt[4]{y_i}}\right \rceil + c \]
	 

\end{document}
